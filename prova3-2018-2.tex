\documentclass[12pt]{article}

\usepackage[utf8]{inputenc}
\usepackage[brazil]{babel}
\usepackage{amsmath,amssymb}
\usepackage{pdfsync}
\usepackage[all]{xy}
\usepackage{color}

\newcommand{\resposta}[1]{ \noindent {\bf Solução}.{\color{blue} #1}}

\title{{\large Universidade de Brasília \\ Instituto de Ciências Exatas \\
Departamento de Ciência da Computação} \\[1cm]
CIC 117536 - Projeto e Análise de Algoritmos \\[.5cm]  Terceira Prova \\[.5cm] Turma: B}
\author{{\bf NP-completude}}
\date{Prof. Flávio L. C. de Moura \\[.5cm] \today}

\begin{document}
\maketitle

\begin{enumerate}
\item {\bf (2.5 pontos)} O problema 2-SAT tem como instâncias as
  fórmulas lógicas formadas por conjunções de disjunções de até dois
  literais, onde um literal é uma variável booleana ou a negação de
  uma variável booleana. Por exemplo, a expressão a seguir é uma
  instância de 2-SAT:

  $$(x_1\lor \neg x_2)\land (\neg x_1 \lor \neg x_3) \land (x_1 \lor x_2) \land x_3$$

  Prove que 2-SAT $\in$ P.

 
  \resposta{
    O problema 2-SAT pode ser resolvido em tempo polinomial fazendo uma transformação da fórmula original para um grafo. Primeiramente é possível observar que uma clausula
    qualquer \(A \lor B\) é equivalente a \(\neg A \implies B\) assim como \(\neg B \implies A\) o que significa dizer que ~A verdadeiro implica em B ser verdadeiro para satisfazer a cláusula. Logo podemos
    construir uma grafo onde cada cláusula do 2-SAT irá adicionar 2 arestas, e cada variável A irá acrescentar 2 vértices \(\neg A\) e A no grafo, o que pode ser feito em tempo polinomial relativo a representação na entrada. Sabendo da construção acima, é possivel mostrar por contradição que caso exista um caminho de uma variável X para \(\neg X\) e um caminho de \(\neg X\) para X então a fórmula não será satisfazivel.

\\ Prova:
	Suponha que existe um caminho X para \(\neg X\) e \(\neg X\) para X no grafo e também uma designação de valores para vada variável da fórmula original para a qual está será verdadeira. Suponha que existe o seguinte caminho:
			
\[X \implies ... \implies A \implies B \implies ... \implies \neg X\]
		
    Sabendo que X é verdadeiro neste caminho entao \(\neg X\) deverá ser falso portando todas as variáveis no intervalo [B,\(\neg X\)] devem ser falsas para que estas clausulas possam ser verdadeiras,da mesma forma para o caminho [X, A] todas as variaveis devem ser verdadeiras para que estas cláusulas também sejam verdadeiras. Porém existe um impasse na cláusula \(A \implies B\) onde B deverá ser falso e A deverá ser verdadeiro o que gera um valor verdade falso para esta cláusula fazendo com que a fórmula não seja satisfazivel para X = v. O mesmo raciocinio pode ser feito para o caminho \(\neg X \implies X\) que implicaria na fórmula original também não ser satisfazivel para um valor de \(\neg X\) = v. Portanto se existem estes dois caminhos não é possivel encontrar um valor verdade para X que satisfaça a fórmula orignal o que contradiz a hipótese de que tal designação existe. Para encontrar um tal caminho é possivel rodar um algortimo de componentes fortemente conexas O(V+E) e verificar se em alguma componente X e \(\neg X\), para alguma variável da fórmula, aparecem.
    
 }
 
\item {\bf (2.5 pontos)} Em aula, assumimos que SAT é um problema
  NP-completo (Teorema de Cook-Levin), e a partir deste fato mostramos
  que 3-SAT e CLIQUE também são problemas NP-completos. As reduções
  foram feitas de acordo com o seguinte diagrama:

  $$\xymatrix{
    SAT \ar[d] \\
    3\mbox{-}SAT \ar[d] \\
    CLIQUE 
  }$$
  
  Um ciclo Hamiltoniano é um ciclo simple que visita cada vértice de
  um grafo exatamente uma vez. Considere o problema de decisão
  HAM-CYCLE que pergunta se um dado grafo (não-dirigido) $G$ possui um
  ciclo Hamiltoniano. Mostre que HAM-CYCLE é um problema
  NP-completo. Sua solução deve ser construída a partir de SAT, 3-SAT
  ou CLIQUE. Caso, você não veja como reduzir diretamente HAM-CYCLE a
  partir destes, mas sabe como fazê-lo a partir de um certo problema
  $Q$ então inicialmente mostre que $Q$ é NP-completo a partir de SAT,
  3-SAT ou CLIQUE, e assim por diante. Digamos que você não saiba como
  mostrar que $Q$ é NP-completo diretamente a partir de SAT, 3-SAT ou
  CLIQUE, mas você sabe como fazê-lo a partir de outro problema $Q'$,
  e também sabe como mostrar que $Q'$ é NP-completo a partir de 3-SAT,
  por exemplo. Então o diagrama correspondente à sua solução seria:

$$\xymatrix{
  SAT \ar[d] & & & \\
  3\mbox{-}SAT \ar[d]\ar[r] & Q' \ar[r] & Q \ar[r] & \mbox{HAM-CYCLE}  \\
  CLIQUE & & & 
}$$

E todas as reduções (de 3-SAT para $Q'$, de $Q'$ para $Q$ e de $Q$ para HAM-CYCLE) devem ser detalhadas na sua solução.

\resposta{
    Escreva aqui sua solução.
  }
  
\item {\bf (2.5 pontos)} Considere o seguinte jogo em um grafo
  (não-dirigido) $G$, que inicialmente contém 0 ou mais bolas de gude
  em seus vértices: um movimento deste jogo consiste em remover duas
  bolas de gude de um vértice $v\in G$, e adicionar uma bola a algum
  vértice adjacente de $v$. Agora, considere o seguinte problema: Dado
  um grafo $G$, e uma função $p(v)$ que retorna o número de bolas de
  gude no vértice $v$, existe uma sequência de movimentos que remove
  todas as bolas de $G$, exceto uma? Mostre que este problema é
  NP-completo. A mesma observação feita no exercício anterior vale
  aqui: a prova deve ser feita a partir de problemas que provamos
  serem NP-completos, e reduções intermediárias, caso existam, devem
  ser incluídas na solução.

  \resposta{
    Escreva aqui sua solução.
  }
  
\item {\bf (2.5 pontos)} Uma fórmula booleana em {\it forma normal conjuntiva com disjunção exclusiva (FNCX)} é uma conjunção de diversas cláusulas, e cada cláusula é uma disjunção exclusiva (XOR) de diversos literais. Lembre-se que a disjunção exclusiva é dada por:

  $$\begin{array}{|l|l|l|}\hline
      a & b & a \oplus b \\ \hline
      V & V & F \\ \hline
      V & F & V \\ \hline
      F & V & V \\ \hline
      F & F & F \\ \hline
  \end{array}$$

  O problema FNCX-SAT pergunta se uma dada fórmula em FNCX é
  satisfatível. Mostre que o problema FNCX-SAT está em $P$, ou então
  que FNCX-SAT é NP-completo. No último caso, a mesma observação feita
  nos dois exercícios anteriores vale aqui: a prova deve ser feita a
  partir de problemas que provamos serem NP-completos, e reduções
  intermediárias, caso existam, devem ser incluídas na solução.

  \resposta{
    Escreva aqui sua solução.
  }
\end{enumerate}

\end{document}

%%% Local Variables:
%%% mode: latex
%%% TeX-master: t
%%% End:
